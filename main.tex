\documentclass[12pt,a4paper,english]{article}
\usepackage[a4paper]{geometry}
\usepackage[utf8]{inputenc}
\usepackage[OT2,T1]{fontenc}
\usepackage[keeplastbox]{flushend}
\usepackage{color}
\usepackage{tikz-cd}
\usepackage{appendix}
\usepackage{babel}
\usepackage{dsfont}
\usepackage{amsmath}
\usepackage{amssymb}
\usepackage{amsthm}
\usepackage{stmaryrd}
\usepackage{color}
\usepackage{array}
\usepackage{hyperref}
\usepackage{graphicx}
\usepackage{mathtools}
\usepackage{natbib}
\usepackage[bb=boondox]{mathalfa}
\geometry{top=3cm,bottom=3cm,left=2.5cm,right=2.5cm}
\setlength\parindent{0pt}
\renewcommand{\baselinestretch}{1.3}

\newcommand\restr[2]{{% we make the whole thing an ordinary symbol
  \left.\kern-\nulldelimiterspace % automatically resize the bar with \right
  #1 % the function
  \vphantom{\big|} % pretend it's a little taller at normal size
  \right|_{#2} % this is the delimiter
  }}
  
% definition of the "structure"
\theoremstyle{plain}
\newtheorem{thm}{Theorem}[section]
\newtheorem{lem}[thm]{Lemma}
\newtheorem{prop}[thm]{Proposition}
\newtheorem{coro}[thm]{Corollary}
\newtheorem{cla}[thm]{Claim}
\theoremstyle{definition}
\newtheorem{conj}{Conjecture}
\newtheorem{defi}{Definition}
\newtheorem*{ex}{Example}
\newtheorem*{rem}{Remark}
\newtheorem{step}{Step}


\title{Master Thesis Sketch}
\date{\today}
\author{Deng Zhiyuan}


\begin{document}
\maketitle
\newpage

\tableofcontents
\newpage

\begin{abstract}
For my master thesis, there will be three parts:
\begin{enumerate}
    \item Recollection of Godement- Jacquet theory and Jacquet-Langlands theory for standard L-functions of $GL_{2}$\cite{godement1974notes,goldfeld2006automorphic}.
    \item A presentation of Gal Dor's results on the equivalence of the space of Zeta integrals for both constructions\cite{dor2020exotic}.
    \item As an extra section, a recollection of Rankin-Selberg integral, and the viewpoint of Sakellaridis uniting all three\cite{sakellaridis2021spherical}.
\end{enumerate}
\subsection*{Godement-Jacquet and Jacquet Langlands}
This part is about the very classic theory based on a lot of calculation. And those calculation only works case by case. They are not systematic method to deal with all the L-functions. 

In Godement-Jacquet theory, they generalized the automorphic representations. The analytic continuation and functional equation of an L-function associated to an automorphic form on $GL(n, \mathbb{A}_{\mathbb{Q}})$ was derived directly from the Poisson summation for $GL(n, \mathbb{A}_{\mathbb{Q}})$ exactly as in Tate's thesis. Godement-Jacquet method is based on the theory of matrix coefficients of automorphic representations. To prove the functional equation is not a present job with a lot of calculation. 

For the Jacquet-Langlands theory, the idea is also from Tate's thesis. The L-functions $L_{\phi}(\chi,s)$ are the local factors used to define global Hecke's L-functions by means of Euler products. Similar as defining this L-functions by Melline transforms, we can define the local factors $L_{\pi}(\chi,s)$ for every irreducible admissible representations $\pi$. Especially, when it's supercuspidal case, the local factors are defined as trivial. As we know, this local factor and the $\epsilon$ can be connected with the representation of Weil group, which is not main story.


\subsection*{Sakellaridis's Theory}

\end{abstract}
\newpage

\vspace{0.5cm}
\section{Classical Theory of Godement-Jacquet and Jacquet-Langlands}
\subsection{Introduction}
Whenever we start to talk about number theory, the legend of L-function which begun with Riemann is always the prototype. Considering L-function, it's related to a lot of famous conjectures in number theory, for example the Riemann conjecture and Ramanujan conjecture. There are three different types of problems at least when one encounter with L-function: the analytic continuation and functional equation; the location of the zeros; and the values at special points.

About 170 years ago, Leonhard Euler first introduce and studied the $\xi$ function over $\mathbb{R}$. Later Bernhard Riemann published article "On the Number of Primes Less Than a Given Magnitude" which extended the Euler definition of Zeta function to $\mathbb{C}$ in 1859 and proved its meromorphic continuation and functional equation, and established a relation between the zeros of $\xi$ functions and the distribution of prime numbers \cite{riemann1859ueber}. In this famous paper, Riemann proposed the Riemann conjecture:
\begin{conj}
All non-trivial zeroes of $\xi(s)$ lie on the line $\mathfrak{Re}(s)=\frac{1}{2}$. 

The Riemann hypothesis is equivalent to the error term in the prime number theorem:
\begin{equation*}
    \#\{\text{Prime numbers}\leq x\}=\int^{x}_{2}\frac{\text{d}t}{\log t}+\text{error term}
\end{equation*}
being as sharp as possible, namely, $O(\sqrt{x}\log x)$ as $x\rightarrow\infty$.
\end{conj}

The analytic theory of L-functions associated to modular forms was studied by Erich Hecke \cite{hecke1936bestimmung}, and later extended to non-holomorphic automorphic forms by Hans Maass \cite{maass1949neue}. The analytic continuation and functional equation of such an L-function was obtained by Mellin transform of an automorphic form $f(x)$:
\begin{equation*}
    f(x)\mapsto \{\mathcal{M}f\}(s)=\phi(s)=\int^{\infty}_{0}x^{s-1}f(x)\text{d}x, 
\end{equation*}
and applying the modular relation $z\mapsto -z^{-1}$ as same as Riemann's proof \cite{riemann1859ueber}. Then the idea was generalized to the Adelic setting by Gelfand-Graev-Pyatetski-Shapiro \cite{gelfand1968representation} and Jacquet-Langlands \cite{langlands1970automorphic}. In 1950, John Tate
developed used a translation invariant integration on the locally compact group of ideles to lift the zeta function twisted by a Hecke character, i.e. a Hecke L-function, of a number field to a zeta integral and study its properties \cite{tate1997fourier}.  By the Poisson summation formula, Tate proved the functional equation and meromorphic continuation of the zeta integral and the Hecke L-function. He also located the poles of the twisted zeta function. During that period, Kenkichi Iwasawa independently developed same results as Tate \cite{iwasawa1992letter}. Hence this theory is often taken as Tate-Iwasawa theory. Iwasawa–Tate theory was extended to the general linear group GL(n) over an algebraic number field and automorphic representations of its adelic group by Roger Godement and Hervé Jacquet in 1972 which formed the foundations of the Langlands correspondence \cite{godement1974notes}. Tate's thesis can be viewed as the $GL(1)$ case of the work by Godement–Jacquet.
On the other side of the story, as same as for $GL_{1}$, Jacquet and Langlands does associate to associate L-function to automorphic from on $GL_{2}$ \cite{langlands1970automorphic}.

The research detailed in this paper was the result of taking a different point
of view on these automorphic L-functions. The idea is to study the entire space
of zeta integrals, instead of just the L-functions that generate them. Gal Dor's main idea is to construct Algebraic structures on automorphic L-functions. As we can find out from the classic theory, L-functions are built as GCDs of classes of zeta integral. Instead of calculation, Gal Dor used theta correspondence to study the equivalence.  The exotic symmetric monoidal
structure of $GL_{2}$-modules enables us to construct an Abelian category of abstractly
automorphic representations, whose irreducible objects are the usual automorphic representations. The category setting he constructed has more conceptual understanding about those two constructions of L-function. For my thesis, I will focus on the first part of his paper, which is the construction of a sysmmetric monoidal structure on the category of smooth representation of $GL_{2}$.

The interesting thing based on Gal Dor is that his method can be generalized. And the paper that we cited here is only part of his Ph.D thesis.
\subsection{Godement-Jacquet Theory}
\subsubsection{Review of Riemann theory}
The most famous L-function is Riemann's $\xi$-function:
\begin{equation*}
    \xi(s)=\sum_{n\geq 1}n^{-s}\ \ \ \ (\mathfrak{Re}s\geq 1).
\end{equation*}

By an easy argument using the fact that every $n\geq 1$ has a unique prime factorisation, one shows that $\xi(s)$ can be written as an Euler product
\begin{equation*}
    \xi(s)=\prod_{p\ prime} (1-p^{-s})^{-1}\ \ \ \ (\mathfrak{Re}s >1).
\end{equation*}
This shows that $\xi(s)$ does not have any zeroes in the region $\mathfrak{Re}s >1$.

We define the completed $\xi$-function by $Z(s)=\pi^{s/2}\Gamma(s/2)\xi(s)$. Here we have used the $\Gamma-$function
\begin{equation*}
    \Gamma(s)=\int^{\infty}_{0}e^{-t}t^{s-1}\text{d}t\ \ \ \ (\mathfrak{Re}s>0). 
\end{equation*}
\begin{rem}
Let's recall a few important properties of $\Gamma$-function:
\begin{enumerate}
    \item The Gamma function satisfies the functional equation $\Gamma(s+1)=s\Gamma(s)$;
    \item The Gamma function has an analytic continuation on $\mathbb{C}$ with simple poles at $s=0,-1,-2,...$
    \item The reciprocal of the Gamma function is entire, i.e. 
    \begin{equation*}
        \frac{1}{\Gamma(s)}=\frac{\sin(\pi s)}{\pi}\Gamma(1-s)
    \end{equation*}
    is entire.
\end{enumerate}
During the proofs of those properties, $\int^{\infty}_{0}e^{-(ax^{2}+b)}=\sqrt{\frac{\pi}{a}}$.
\end{rem}
\begin{lem}\label{PoiR}
Given a Schwazrtz function $f$, and $\hat{f}$ is the Fourier transform, then
\begin{equation*}
    \sum_{n\in\mathbb{Z}}f(n)=\sum_{m\in\mathbb{Z}}\hat{f}(m).
\end{equation*}
\end{lem}
\begin{proof}
Denote $F(x)=\sum_{n\in\mathbb{Z}}f(x+n)$, which is 1-periodic, then we can calculate the Fourier coefficients: 
\begin{equation*}
    \hat{F}(k)=\int^{1}_{0}F(x)e^{-2\pi ikx}\text{d}x=\int^{1}_{0}\sum_{n\in\mathbb{Z}}f(x+n)e^{-2\pi ikx}\text{d}x.
\end{equation*}
Because $f$ is Schwartz, so this integral converges uniformly.
\begin{equation*}
    =\sum_{n\mathbb{Z}}\int^{n+1}_{n}f(x)e^{-2\pi ikx}=\int_{\mathbb{R}}f(x)e^{-2\pi ikx}\text{d}x=\hat{f}(k),
\end{equation*}
by the definition of Fourier series of $f$, $F(x)=\sum\limits_{k\in\mathbb{Z}}\hat{f}(k)e^{ikx}$. Choosing $x=0$, we get the formula as desired:
\begin{equation*}
    \sum_{n\in\mathbb{Z}}f(n)=\sum_{k\in\mathbb{Z}}\hat{f}(k).
\end{equation*}
\end{proof}
\begin{defi}
The Jacobi theta function, or called as the classical theta function is the on right half plane defined by 
\begin{equation*}
    \theta(s)=\sum_{n\in\mathbb{Z}}e^{-\pi n^{2}s},\ \ \ \ \mathfrak{Re}s>0.
\end{equation*}
The theta function here is a holomorphic function on the right half plane.
\end{defi}
\begin{lem}
By the Poisson formula \ref{PoiR}, one can prove that $\theta(s)$ satisfies the functional equation
\begin{equation*}
\theta(\frac{1}{s})=\sqrt{s}\theta(s).
\end{equation*}
\end{lem}

\begin{proof}
Given $g(x)=e^{-\pi tx^{2}}$, in order to get the formula, we need to calculate its Fourier transform of $g(x)$, 
\begin{equation*}
    \hat{g}(t)=\int^{+\infty}_{-\infty}e^{-\pi tx^{2}}\cdot e^{-2\pi iyx}\text{d}x=e^{\pi}(\frac{iy}{\sqrt{t}})\int^{+\infty}_{-\infty}e^{-\pi(\sqrt{t}x+\frac{ix}{\sqrt{t}})^{2}}\text{d}x=\frac{1}{\sqrt{t}} e^{-\pi \frac{y^{2}}{t}}
\end{equation*}
Then 
\begin{equation*}
    \theta(s)=\sum_{n\in\mathbb{Z}}e^{-\pi n^{2}s}=\sum_{n\in\mathbb{Z}}g(n)=\sum_{m\in\mathbb{Z}}\hat{g}(m)=\frac{1}{\sqrt{s}}\sum_{m\in\mathbb{Z}}e^{-\pi \frac{n^{2}}{s}}=\frac{1}{\sqrt{s}}\theta(\frac{1}{s}).
\end{equation*}
\end{proof}
Before we moved to the proof of the analytic continuation of Riemann Zeta function, we need to know the behaviour of the theta functions near zero and infinity to justify the Mellin transform later in proof.

For theta function, there exists a constant $C=\pi-1> 0$ such that for all sufficiently small $t>0$ the following inequality holds: 
\begin{equation*}
    |\theta(t)-t^{\frac{1}{2}}|< e^{-C/t}.
\end{equation*}

Usually the Mellin transform is that Given a function $f(t)$ with $\mathfrak{Re}t>0$, its Mellin transform is defined to be
\begin{equation*}
    \mathcal{M}_{f}(s):=\int^{\infty}_{0}f(t)t^{s-1}\text{d}t,
\end{equation*}
which is defined for all values $s$ for which the integral exists.
\begin{thm}
The function $Z(s)$ can be continued to a meromorphic function on the whole complex plane with a simple pole at $s=1$ with residue 1, a simple pole at $s=0$ with residue -1, and no other poles. It satisfies the functional equation 
\begin{equation*}
        Z(s)=Z(1-s)
\end{equation*}
i.e., $\xi(s)$ satisfies the functional equation
\begin{equation*}        \pi^{-1/2}\Gamma(\frac{s}{2})\xi(s)=\pi^{(1-s)/2}\Gamma(\frac{1-s}{2})\xi(1-s).
\end{equation*}
\end{thm}
\begin{proof}
We express $Z(s)$ as the Mellin transform of a modular form. The modular form we need is Jacobin theta-function. But theta function behaves like $t^{-1/2}$ at 0 and converges rapidly to 1 at infinity. So we need to introduce some correction term to make the integral converge. So We now consider the Mellin transform of $\theta$, defined as \begin{equation*}
    \mathcal{M}_{\theta}(s)=\int^{\infty}_{1}(\theta(t)-1)t^{s-1}\text{d}t+\int^{1}_{0}(\theta(t)-\frac{1}{\sqrt{t}})t^{s-1}\text{d}t,\ \ \ \ (\mathfrak{Re}> 1).
\end{equation*}
From this we get the following equation, due to Riemann: $Z(s)=\frac{1}{2}\mathcal{M}_{\theta}(s/2)$. And we can rewrite $\mathcal{M}_{\theta}(s/2)$ as follows.  We obtain, for $\mathfrak{Re}s>1$, 
\begin{align*}
\int^{1}_{0}\bigg(\theta(t)-\frac{1}{\sqrt{t}}\bigg)t^{\frac{s}{2}-1}\text{d}t
&= \int^{1}_{0}\theta(t)t^{\frac{s}{2}-1}\text{d}t-\int^{1}_{0}t^{\frac{s-1}{2}}\text{d}t\\
&= \int^{1}_{0}\bigg(\sum_{n\in\mathbb{Z}}e^{-\pi n^{2}t}\bigg)t^{\frac{s}{2}-1}\text{d}t-\frac{2}{s-1}\\
&= \int^{1}_{0}t^{\frac{s}{2}-1}\text{d}t+2\int^{1}_{0}\bigg(\sum_{n\geq 1}e^{-\pi n^{2}t}\bigg)t^{\frac{s}{2}-1}\text{d}t +\frac{2}{1-s}\\
&= 2\sum_{n\geq 1}\int^{1}_{0}e^{-\pi n^{2}t}t^{\frac{s}{2}-1}\text{d}t+\frac{s}{2}+\frac{2}{1-s}
\end{align*}
Therefore we have
\begin{equation*}
    \mathcal{M}_{\theta}(s)=2\sum_{n\geq1}\int^{\infty}_{0}e^{-\pi n^{2}t}t^{\frac{s}{2}-1}\text{d}t+\frac{2}{s}+\frac{2}{1-s}
\end{equation*}
By the substitution $t\mapsto 1/t$ we get
\begin{equation*}
    \frac{1}{2}\mathcal{M}_{\theta}(s)=\pi^{-\frac{s}{2}}\Gamma(\frac{s}{2})\xi(s)+\frac{1}{s}+\frac{1}{1-s}.
\end{equation*}
Both integrals in the definition of $\mathcal{M}_{\theta}(s)$ converge, it's an entire function. Then we can define an analytic continuation of the zeta function on the whole complex plane, which agrees with $\xi(s)$ for $\mathfrak{Re}s>1$, by 
\begin{equation*}
    \xi(s)=\frac{\pi^{\frac{s}{2}}}{\Gamma(\frac{s}{2})}\bigg(\frac{1}{2}\mathcal{M}_{\theta}(s)-\frac{1}{s}-\frac{1}{1-s}\bigg),
\end{equation*}
whose poles are at $s=0$ or $s=1$ as $\mathcal{M}_{\theta}(s) $ and $\frac{1}{\Gamma(s)}$ are entire. We can replace $s\Gamma(\frac{s}{2})$ by $\frac{2}{s/2}\Gamma(s/2)=2\Gamma(s/2+1)$ which converges to $2\Gamma(1)$ as $s$ goes to zero. Hence it's nonzero and the only pole is at 1. We now compute the residue:
\begin{equation*}
    \lim_{s\mapsto 1}(s-1)\frac{\pi^{s/2}}{\Gamma(s/2)}\bigg(\frac{1}{2}\mathcal{M}_{\theta}(s)-\frac{1}{s}-\frac{1}{1-s}\bigg)=\frac{\pi^{1/2}}{\Gamma(1/2)}=1.
\end{equation*}
Now the only thing left is $Z(s)=Z(1-s)$. Since $Z(s)=\frac{1}{2}\mathcal{M}_{\theta}(s)-\frac{1}{s}-\frac{1}{1-s}$, which means we only need to check that $\mathcal{M}_{\theta}(s)=\mathcal{M}_{\theta}(1-s)$, and we need to use $y=\frac{1}{t}$:
\begin{align*}
        \mathcal{M}_{\theta}(s)=
        &= \int^{1}_{0}y^{-s/2-1}(\theta(1/y)-1)\text{d}y+\int^{\infty}_{1}y^{-s/2-1}(\theta(1/y)-\sqrt{y})\text{d}y\\
        &= \int^{1}_{0}y^{(-1/2-s/2)}(\theta(y)-\frac{1}{\sqrt{y}})\text{d}y+\int^{\infty}_{1}y^{-1/2-s/2}(\theta(y)-\sqrt{y})\text{d}y\\
        &=\mathcal{M}_{\theta}(1-s).
\end{align*}
Now the proof iscomplete.
\end{proof}
\subsubsection{Review of Tate's thesis}
By choosing an appropriate set of test functions and an appropriate measures, we can extend Riemann zeta to an idelic integral.
For the set of test functions are defined as following:
\begin{defi}
Adelic Bruhat-Schwartz functions are linear combinations of functions $\Phi: \mathbb{A}_{\mathbb{Q}}\rightarrow \mathbb{C}$ where $\Phi=\prod\limits_{v\leq\infty} \Phi_{v}$, in which $Phi_{v}$ are the characteristic function of $\mathbb{Z}_{v}$ for all but finitely many $v< \infty$. And $\Phi_{\infty}$ is Schwartz on $\mathbb{R}$, $\Phi_{v}$ is Bruhat-Schwartz function on $\mathbb{Q}_{v}$. The space of Bruhat-Schwartz functions are denoted as $S(\mathbb{A_{\mathbb{Q}}})$.
\end{defi}
Recall the definition of Schwartz on $\mathbb{R}$:
\begin{equation*}
    S(\mathbb{R},\mathbb{C})=\{f\in C^{\infty}(\mathbb{R},\mathbb{C})|\forall \alpha,\beta\in \mathbb{N}, ||f||_{\alpha,\beta}=\text{sup}_{x\in\mathbb{R}}x^{\alpha}(D^{\beta}f)<\infty\}
\end{equation*}
Now let's give the defition of Idelic integral: 
\begin{defi}
Define the idelic integral for factorizable idelic functions $\Phi=\prod_{v}\Phi_{v}$ such that $\Phi_{p}$ is the characteristic function $\mathbb{1}_{\mathbb{Z}_{p}}$ for almost all primes $p$ by 
\begin{equation*}
    \int_{\mathbb{A}^{\times}_{\mathbb{Q}}}\Phi(s)\text{d}^{\times}x=\prod_{v\in S}\int_{\mathbb{Q}^{\times}_{v}}\Phi_{v}(x_{v})\text{d}^{\times}x_{v},
\end{equation*}
in which $S$ is a finite set containing $\infty$ and all primes that $\Phi_{p}\not=\mathbb{1}_{\mathbb{Z}_{p}}$. The measure $\text{d}^{\times}x$ is defined as following:
\begin{equation*}
\text{d}^{\times}x= \left\{
    \begin{array}{cc}
        \frac{\text{d}x_{\infty}}{|x_{\infty}|_{\infty}}, &  \text{if}\ v=\infty\\
        \frac{1}{1-p^{-1}}, & \text{if}\ v=p, p\ \text{are finite primes}
    \end{array}
    \right
\end{equation*}
Such that $\text{d}^{\times}x_{p}$ is normalized so that $\int_{\mathbb{Z}^{\times}}\text{d}^{\times}x_{p}=1$.
\end{defi}
\begin{defi}
Idelic absolute value: $x=\{x_{\infty},x_{2},...\}\in\mathbb{A}^{\times}_{\mathbb{Q}}$, then $|x|_{\mathbb{A}}=\prod_{v\geq\infty}|x_{v}|_{v}$.
\end{defi}
In order to get idelic absolute value, we choose a special test function: for $x=\{x_{\infty},x_{2},x_{3},...\}$, then
\begin{equation*}
    h(x)=e^{-\pi x_{\infty}^{2}}\prod_{v<\infty}\mathbb{1}_{\mathbb{Z}_{v}}(x_{v})\in S(\mathbb{A}_{\mathbb{Q}}).
\end{equation*}
After some calculation, we know that the Fourier transform of $h(x)$ is equal to $h(x)$.

Now we are ready to write down the zeta integral:
\begin{equation*}
    Z(s)=\int_{\mathbb{A}^{\times}_{\mathbb{Q}}}h(x)|x|^{s}_{\mathbb{A}_{\mathbb{Q}}}\text{d}^{\times}x, \ \ s\in\mathbb{C},\ \mathfrak{Re}(s)>1.
\end{equation*}
For every automorpic form $\phi$ on $GL_{1}\mathbb{A}_{\mathbb{Q}}$ can be uniquely expressed in the form 
\begin{equation*}
    \phi(g)=c\cdot \chi_{idelic}(g)\cdot |g|^{it}_{\mathbb{A}},\ \ \forall g\in\mathbb{A^{\times}_{Q}},\ 
\end{equation*}
in which $c\in\mathbb{C},\ t\in\mathbb{R}$ are fixed constants. And the character $\chi_{idelic}$ is an idelic lift of a Dirichilet character $\chi(mod\ p^{f})$. The idelic lift of $x$ is the Hecke character $\chi_{idelic}:\mathbb{Q}^{\times}\backslash\mathbb{A}^{\times}_{\mathbb{Q}}\rightarrow\mathbb{C}$ defined as 
\begin{equation*}
    \chi_{idelic}(g)=\chi_{\infty}(g_{\infty})\cdot\chi_{2}(g_{2})\cdot\chi_{3}(g_{3})\cdot\cdot\cdot\ \ \ g=\{g_{\infty},g_{2},g_{3},...\}\in\mathbb{A^{\times}_{Q}},
\end{equation*}
where 
\begin{equation*}
x_{\infty}(g_{\infty})=\left\{
\begin{array}{ccc}
     1& \chi(-1)=1 \\
     1&  \chi(-1)=-1\ \text{if}\  g_{\infty}>0\\ 
     -1& \chi(-1)=-1\ \text{if}\ g_{\infty}<0
\end{array}
\right
\end{equation*}
and 
\begin{equation*}
    x_{v}(g_{v})=\left\{
    \begin{array}{cc}
        \chi(v)^{m} &\text{if}\ g_{v}\in v^{m}\mathbb{Z}^{\times}_{v}\ \text{and}\  v\not=p  \\
         \chi(g)^{-1} & \text{if}\ g_{v}\in p^{k}(j+p^{f}\mathbb{Z}_{p})\ \text{with}\ j,k\in\mathbb{Z}.\ (j,p)=1,\ \text{and}\ v=p 
    \end{array}
    \right
\end{equation*}
Let's write down adelic Poisson summation formula:
\begin{equation}
    1+\sum_{\alpha\in\mathbb{Q}^{\times}}h(\alpha x)=\frac{1}{|x|_{\mathbb{A}}}+\frac{1}{|x|_{\mathbb{A}}}\sum_{\alpha\in\mathbb{Q}^{\times}}h(\frac{\alpha}{x})
\end{equation}
\subsubsection{Godement-Jacquet}
Tate's method is generalized to L-functions of $GL_{n}$ for any $n\geq1$ by Roger Godement and Herv\'{e} jacquet \cite{goldfeld2006automorphic}. Godement and Jacquet considered the inner forms of $GL_{n}$ and the affine emmbedding $\mathbb{G}_{m}\rightarrow\mathbb{G}_{a}$ is replaced as $GL_{n}\rightarrow M_{n}$.

\textcolor{red}{XXXXXXXXXXXXXXXXXXXXXXXXXXXXXXXXXXXXXXXXXXXXXXXXXXXXXXXXXXXXXXXX}
\subsection{Jacquet-Langlands Theory}
\subsubsection{Whittaker Model and Kirillov Model}
First let's review the Whittaker model and Kirillov model. 
\begin{defi}
representation $(\pi, V)$ is  generic if it has a Whittaker model. 
\end{defi}
Given $V$ smooth irreducible representation of $G(F)$, A linear functional $\lambda: V\rightarrow \mathbb{C}$ is called a Whittaker functional for $\chi$ if for all $u\in U(F)$ and $\xi\in V$ we have $\lambda(u\circ\xi)=\chi(u)\lambda(\xi)$. Fix nonzero Whittaker functional $\lambda$, and for $v\in V$, define a function $w_{v}: G(F)\rightarrow \mathbb{C}$ by $w_{v}(g)=\lambda(gv)$. Denote the space of such functions by $W=W_{\lambda}$, $W$ is closed under addition and scalar multiplication if $gW_{v}=W_{gv}$. The representation $W$ is called a Whittaker model up to $G(F)-$isomorphism.

Next is the Kirillov model, the goal of Kirillov model is that every admissible irreducible representation of $GL_{2}(F)$ can be taken as a space of function space over $F^{\times}$. And every finite irreducible admissible representation of $G(F)$ is 1-dimensional and $\pi(g)=\chi(\text{det}(g))$, in which $\chi$ is a character of $F^{\times}$ and $\pi(g)=\chi(\text{det}(g))$. If $|\chi(x)|=1$, then call it unitary. 

For the every finite dimensional smooth representation of $GL_{2}(F)$, it's just one dimensional. Given $\forall g\in GL_{2}(F)$, we can write $g=\begin{pmatrix}
\text{det}(g) & 0\\
0 & 1
\end{pmatrix}\cdot  g'$,\ $g'\in SL_{2}(F)$. So we only need to construct infinite-dimensional representation. 
\begin{thm}
Given $(\pi, V)$ infinite-dimensional irreducible representation of $GL_{2}(F)$, there exists one and only one $V'$, which is a complex function space over $F^{\times}$, the representation $\pi'$ of $GL_{2}(F)$ Satisfies the following conditions:
\begin{enumerate}
    \item The $\pi'$ and $\pi$ are equivalent;
    \item $\pi'\begin{pmatrix}
    a& b\\
    0&1
    \end{pmatrix}\xi'(x)=\tau_{F}(bx)\xi'(ax)$, in which $a, x\in F^{\times},\ b\in F,\ \forall\xi'\in V'$.
\end{enumerate}
Every $\xi\in V'$ is locally constant and vanishes at compact subset of $F$. All Locally constant functions that vanishes outside of compact subsets belong to $V'$. And we denote the space such functions by $\psi(F^{\times})$ and the $codim(\psi(F^{\times}))$ is finite. 

\end{thm}
Then $\pi'$ will be called a Kirillov representation of $GL_{2}(F)$ or Kirillov model of the corresponding class of irreducible representation.

There is an isomorphism between $V$ and $V'$, this isomorphism is compatible with $\pi$ and $\pi'$. Given a linear map $L$ of $V$, it maps $\xi$ to $\xi'(1)$. And it satisfies that
\begin{equation*}
    L\bigg[\pi\begin{pmatrix}
    1 & b\\
    0& 1
    \end{pmatrix}\xi\bigg]=\tau_{F}(b)L(\xi),\ \forall\xi\in V,\ b\in F. 
\end{equation*}
And 
\begin{equation*}
    \xi'(x)=L\bigg[\pi\begin{pmatrix}
    x & 0\\
    0&1
    \end{pmatrix}\xi\bigg].
\end{equation*}
\subsubsection{Hecke L-function}
This section is a recollection of Hecke L-function over $GL_{1}(\mathbb{A})$, which is the work of Hecke, as well as Tate's thesis \cite{tate1997fourier}. We follow chapter 6 of Gelbart's book \cite{gelbart2016automorphic}.

In this section, let $F$ be a number field and $\mathbb{A}_{F}$ its ring of adeles. An automorphic representation of $GL_{1}(F)$ is any irreducible unitary representation of $GL_{1}(\mathbb{A}_{F})$ appearing in $L^{2}(GL_{1}(F)\backslash GL_{1}(\mathbb{A}_{F}))$. As a $GL_{1}(\mathbb{A}_{F})$-module,
\begin{equation*}
    L^{2}(GL_{1}(F)\backslash GL_{1}(\mathbb{A}_{F}))=\int^{\oplus}_{GL_{1}(F)\backslash GL_{1}(\mathbb{A}_{F})}\psi(g)
\end{equation*}
so an automorphic representation is simply a grossencharacter of $F$, i.e.  a character $\psi:F^{\times}\backslash \mathbb{A}_{F}^{\times}\rightarrow S^{1}$. Thus the theory we are discussing is that in the case of $GL_{1}$, simply Hecke's theory of L-functions associated to grossencharacters. 

Let $\psi: F^{\times}\backslash \mathbb{A}_{F}^{\times}\rightarrow S^{1}$ be a grossencharacter of $F$. We can decompose it $\psi=\prod_{v}\psi_{v}$ as a product of characters $\psi_{v}$ on $F^{\times}$, with $\psi_{v}$ unramified for almost all $v$. Let $S$ be the set of ramified places. For every $v\not\in S$, define $\chi(v)=\psi_{v}(\pi_{v})$ where $\pi_{v}\in F_{v}$ is a uniformizer. Note that the choice of uniformizer does not matter. We can extend $\chi$ by multiplicativity to the set of ideals of $F$ prime to $S$. 

\begin{defi}
The L-series associated to $\phi$, or to $\chi$, is 
\begin{equation*}
    L(s, \chi)=\sum_{S\nmid\mathfrak{a}}\frac{\chi(\mathfrak{a})}{N(\mathfrak{a})^{s}}=\prod_{v\not\in S}\bigg(1-\frac{\chi(v)}{N(v)^{s}}\bigg)^{-1}.
\end{equation*}
\end{defi}
\begin{thm}
$L(s,\chi)$ has nice L-function properties:
\begin{enumerate}
    \item It converges for $\mathfrak{Re}(s)>1$;
    \item It has a meromorphic continuation to the whole plane, with a simple pole at $s=1$ if $\chi$ is trivial and no poles otherwise;
    \item There is a constant $A$, a constant $W(\chi)$ of modulus 1, and a gamma factor $\Gamma(s,\chi)$ such that $\mathfrak{L}(s,\chi)=s(s-1)A^{s}\Gamma(s,\chi)L(s,\chi)$ is entire and satisfies the functional equation
    \begin{equation*}
        \mathfrak{L}(1-s,\chi^{-1})=W(\chi)\mathfrak{L}(s,\chi)
    \end{equation*}
\end{enumerate}
\end{thm}
We skip the proof of this theorem. For the detail of this theorem, check in \cite{gelbart2016automorphic}.
\subsubsection{Jacquet-Langlands Zeta Integral and L-function\cite{langlands1970automorphic}}
Just as  for $GL_{1}$, we now want to associate L-functions to automorphic forms for $GL_{2}$. Before we discuss Jacquet-Langlands, let's review the method for classical holomorphic cusp forms: if $f\in S_{k}(N, \psi)$ has Fourier expansion $f=\sum_{n}a_{n}q^{n}$ at infinity, then we can define an L-function by 
\begin{equation*}
    L(f,s)=\frac{1}{(2\pi^{s})}\Gamma(s)\sum_{n\geq 1}\frac{a_{n}}{n^{s}}=\int^{\infty}_{0}f(iy)y^{s-1}
    \text{d}y.
\end{equation*}
That is, by taking the Mellin transform of $f$ along the vertical half-line $\{iy:y>0\}$. To see how to define our L-functions more generally we rewrite this in the adelic setting.

Let $\phi_{f}(g)=f(g_{\infty}(i))j(g_{\infty},i)^{-k}\psi(k_{0})$ in which $g=\gamma g_{\infty}k_{0}\in GL_{\mathbb{A}}\in G_{\mathbb{Q}}G^{+}_{\infty}K_{0}$ be the adelic automorphic form associated to $f$. For simplicity suppose $N=1$ and $\psi$ is trivial. Then $\phi_{f}(g)$ is right $K_{0}-$invariant, and left $GL_{2}(\mathbb{Q})-$invariant. From the definition of $\phi_{f}$ we see, for real $y>0$,
\begin{equation*}
    \phi_{f}\bigg(\begin{pmatrix}
    y&0\\
    0&1
    \end{pmatrix}\bigg)=f(iy).
\end{equation*}
Thus
\begin{equation*}
    L(f,s)=\int_{\mathbb{Q^{\times}}\backslash\mathbb{A}^{\times}}\phi_{f}\bigg(\begin{pmatrix}
    y&0\\
    0&1
    \end{pmatrix}\bigg) |y|^{s}\text{d}^{\times}y
\end{equation*}

Now we get Fourier analysis involved. Recall that the characters of $\mathbb{Q}^{\times}\backslash\mathbb{A}^{\times}$ are given by $\tau(\lambda x)$ for various $\lambda\in\mathbb{Q}^{\times}$, where $\tau(x)=\prod_{p\leq \infty}\tau_{p}(x)$ and $\tau_{\infty}(x)=e^{2\pi ix_{\infty}}$ and $\tau_{p}(x)=1$ if and only if $x_{p}\in \mathcal{O}_{p}$. Thus $    \phi_{f}\bigg(\begin{pmatrix}
    1&x\\
    0&1
    \end{pmatrix}g\bigg)$ has a Fourier expansion as a function of $x$:
\begin{equation*}
    \phi_{f}\bigg(\begin{pmatrix}
    1&x\\
    0&1
    \end{pmatrix}\bigg)=\sum_{\lambda\in\mathbb{Q}}\phi_{f,\lambda}(g)\tau(\lambda x),
\end{equation*}
where 
\begin{equation*}
    \phi_{f,\lambda}(g)=\int_{\mathbb{Q}^{\times}\backslash\mathbb{A}^{\times}}\phi_{f}\bigg(\begin{pmatrix}
    1&x\\
    0&1
    \end{pmatrix}g\bigg)\overline{\tau\lambda x}\text{d}x
\end{equation*}
is the $\lambda-$th Fourier coefficient of $\phi_{f}\bigg(\begin{pmatrix}
    1&x\\
    0&1
    \end{pmatrix}g\bigg)$ depending on $g$.

Suppose $f\in S_{k}(\Gamma(1))$ has Fourier expansion $f=\sum_{n}a_{n}e^{2\pi i n\pi}$, and $\phi_{f}$ is the associated adelic automorphic form. Then for real $y>0$, 
\begin{equation*}
    \phi_{f}\bigg(\begin{pmatrix}
    y&0\\
    0&1
    \end{pmatrix}\bigg)=a_{n}e^{-2\pi ny}\ \ \text{if}\ \lambda=n\in\mathbb{Z}
\end{equation*}
otherwise it's 0.
Letting $x=0$, so that $\begin{pmatrix}
1&x\\
0&1
\end{pmatrix}=\text{Id}$ and $\tau(\lambda x)=1$, and defining 
\begin{equation*}
    W_{\phi_{f}}(g)=\int_{\mathbb{Q}\backslash\mathbb{A}}\phi\bigg(\begin{pmatrix}
    1&x\\
    0&1
    \end{pmatrix}g\bigg)\overline{\tau(x)}\text{d}x
\end{equation*}
to be the first Fourier coefficient of $\phi_{f}$, we find 
\begin{equation*}
    \phi_{f}\bigg(\begin{pmatrix}
    y&0\\
    0&1
    \end{pmatrix}\bigg)=\sum_{\lambda\in\mathbb{Q}^{\times}}\phi_{f,\lambda}\bigg(\begin{pmatrix}
    y&0\\
    0&1
    \end{pmatrix}\bigg)=\sum_{\lambda\in\mathbb{Q}^{\times}}W_{\phi_{f}}\bigg(\begin{pmatrix}
    \lambda y&0\\
    0&1
    \end{pmatrix}\bigg)
\end{equation*}
Now our L-functions becomes
\begin{equation*}
    L(f,s)=\int_{\mathbb{Q}^{\times}\backslash\mathbb{A}^{\times}}\sum_{\lambda\in\mathbb{Q}^{\times}}W_{\phi_{f}}\bigg(\begin{pmatrix}
    \lambda y&0\\
    0&1
    \end{pmatrix}\bigg)|y|^{s}\text{d}^{\times}y=\int_{\mathbb{A}^{\times}}W_{\phi_{f}}\bigg(\begin{pmatrix}
    y&0\\
    0&1
    \end{pmatrix}\bigg)|y|^{s}\text{d}^{\times}y.
\end{equation*}
In other words, $L(f,s)$ is the adelic Mellin transform along $\begin{pmatrix}
y&0\\
0&1
\end{pmatrix}$ of the first Fourier coefficient of $\phi_{f}$. This is supposed to suggest that in general, the L-function associated to an automorphic representation $\pi$ should be the Mellin transform of the first Fourier coefficient of some distinguished function in the space of $\pi$.

As it is described in this last section, the function $W_{\phi_{f}}$ Satisfies the conditions of Whittaker model: the right-translates of $W_{\phi_{f}}$ generate a space $W_{\pi_{f}}$ of functions $W$ on $GL_{2}(\mathbb{\mathbb{A}})$. The equivalent representation $W(\pi_{f})$ is called the Whittaker model of $\pi_{f}$.

To every function in the local Whittaker model we associate a $\xi-$function, a special one of which will be our local L-function. Let $\pi_{v}$ be an irreducible admissible representation of $GL_{2}(F_{v})$, $\chi$ a unitary character of $F^{\times}_{v}$, $g\in GL_{2}F_{v}$, $W\in W(\pi_{v})$. Then define a local $\xi-$function for all of this date by
\begin{equation*}
    \xi(g,\chi,W,s)=\int_{F^{\times}_{v}}W\bigg(\begin{pmatrix}
    a&0\\
    0&1
    \end{pmatrix}\bigg)\chi(a)|a|^{s-1/2}\text{d}^{\times}a.
\end{equation*}
\begin{thm}
\begin{enumerate}
    \item The integral defining $\xi(g,\chi,W,s)$ converges in some right half plane;
    \item There is a $W^{0}\in W(\pi_{v})$ such that $L(\chi\otimes \pi_{v},s)=\xi(1,\chi,W^{0},s)$ is an Euler factor making $\frac{\xi(g,\chi,W,s)}{L(\chi\otimes\pi_{v},s)}$ entire for every $g, chi, W$. ;
    \item $\xi(W,s)$ has an analytic continuation the whole plane satisfying the functional equation
    \begin{equation*}
        \frac{\xi(g,\chi,W,s)}{L(\chi\otimes\pi_{v},s)}\varepsilon(\pi_{v},\chi,s)=\frac{\xi(wg,\chi^{-1}\psi^{-1},W,1-s)}{L(\chi^{-1}\psi_{v}^{-1}\otimes\pi_{v},1-s)}
    \end{equation*}
    for some function $\varepsilon(\pi_{v},\chi,s)$ independent of $g, W$, where $w=\begin{pmatrix}
    0&1\\
    -1&0
    \end{pmatrix}$, $\phi_{v}$ is the central character of $\pi_{v}$.
\end{enumerate}
\end{thm}
For a finite place $v$, Euler factor means $\frac{1}{P(q^{s})}$ where $P$ is a polynomial with constant term 1 and $q=|w_{v}|$, for an infinite place, it means some kind of $\Gamma$-factor. We skip the details of this theorem \cite{gelbart2016automorphic}. 
\section{Exotic Monoidal Structures by Gal Dor}
\subsection{Introduction}
The goal of this section is to use theta correspondence to understand the equivalence between Godement-Jacquet Zeta integral and Jacquet-Langlans Zeta integral. Then by the relation of Zeta integral, we will be able to understand the equivalence of corresponding L-functions in a more conceptual way rather than cumbersome calculation. Here we only consider $Gl_{2}(F)$, in which $F$ is a non-Archimedean local field and the characteristic of $F$ is not equal 2. But the method we used is available for $GL_{n}$ or global situation, which won't be covered in this thesis.

To construct the Godement-jacquet Zeta integral, there should be a test function from Bruhat-Schwartz function and a matrix coefficient from $\tilde{V}\otimes V$, in which $(\pi, V)$ is a representation of $G=GL_{2}(F)$ and $\tilde{V}$ is the contragredient representation of $V$. Putting all the ingredient for Godement-Jacquet Zeta integral, we denote it as $\tilde{V}\otimes_{G}Y\otimes_{G}V$. On the other side, the Jacquet-Langlands only need one ingredient: a vector from Kirillov model of $(\pi, V)$. Its ingredient saves in $\mathcal{K}(V(-1)(1)$. So the dream here is to construct an isomorphism between those two ingredient space of Godement-jacquet Zeta integral and Jacquet-Langlands Zeta integral with the respect of GCD procedure of constructing corresponding L-functions. Evidently, the Kirillov model can be taken as a $G-$model so we need to make $\tilde{V}\otimes_{G}Y\otimes_{G} V$ into a $G$-module by Schr\"{o}dinger model as well. So the isomorphism of $G$-modules that are relevant to the Zeta integral is a categorification of the  equivalence between of Godement-Jacquet method and Jacquet-Langlands method. The relative category of $G-$modules have symmetric monoidal structure. But the topic of the monoidal structure on the category of $G$-module is not the main story of this thesis. But we will try to show the basic idea of monoidal structure that we desire in the appendix.
\subsection{Recollection of Weil Representation}
\subsubsection{Weil Representation}
Introduce the notation first, $F$ as a non-Archimedean local field, in order to avoid technical problem, we require that characteristic of $F$ is not equal 2. In this section we require that all representations $(\pi, V)$ are smooth and complex. And we require that $V$ are all finite dimensional. Denote the space of all locally constant and compact supported functions by $\mathcal{S}(V)$.

\begin{defi}
Given $W$ a finite dimensional vector space over $F$ with a non-degenerate alternating form $<, >$. Here, the word alternating means that $<x,x>=0,\ \forall x\in W$, nondegenerate means that if for all $y$ we have $<x, y>=0$, then $x$ must be zero. For convenience, we define that the dimension of $W$ is $2n$. Then the $W$ is the symplectic space.
\end{defi}
\begin{defi}
Heisenberg group $H(W)$ is non-trivial central extension of $W$ by $F$, so $H(W)$ is a group of pairs as
\begin{equation*}
    \big\{ (w,t): w\in W, t\in F \big\},
\end{equation*}
and the operation of the group is as follow:
\begin{equation*}
    (w_{1},t_{1})(w_{2},t_{2})=(w_{1}+w_{2},t_{1}+t_{2}+\frac{1}{2}<w_{1},w_{2}>).
\end{equation*}
\end{defi}
\begin{rem}
We get a short exact sequence:
\begin{equation*}
   0\rightarrow F\rightarrow H(W)\rightarrow W\rightarrow 0 
\end{equation*}
$F$ is cmmutator subgroup of Heisenberg group.
\end{rem} 

Next in order to consider the representation of Heisenberg group, all the finite dimensional representation are one dimensional, and factor through $W$. So the next task is to construct infinite dimensional representation:

First we break $W$ into $W_{1}\bigoplus W_{2}$, in which $W_{1}$ and $W_{2}$ are called the maximal totally isotropic subspace or called the Lagrangian subspace. This is the complete polarisation of $W$.

Given an additive character $\phi$ of $F$, the Schr\"odinger representation of Heisenberg group $\rho_{\phi}: H(W)\rightarrow \mathcal{S}(W_{1})$ is defined as follow: 
\begin{enumerate}
    \item $\rho_{\phi}(w_{1})f(x)=f(x+w_{1}),\ x,w_{1}\in W_{1}$;
    \item $\rho_{\phi}(w_{2})f(x)=\phi(<x,w_{2}>)f(x),\ x\in W_{1},\ w_{2}\in W_{2}$;
    \item $\rho_{\phi}(t)f(x)=\phi(t)f(x),\ t\in F,\ x\in W_{1}$.
\end{enumerate}
By those properties, it defines smooth representation of $H(W)$, which is called Schr\"odinger representation.

The most important result regrading Weil representation, it's the following theorem:
\begin{thm}
\textbf{Stone, Van Neuman}: Heisenberg group $H(W)$ has a unique irreducible smooth representation on which $F$ operates via character $\phi$. Denote the unique irreducible smooth representation of $H(W)$ by $(\rho_{\phi}, \mathcal{S})$. Here we skip the proof of this theorem.
\end{thm}

Next goal is to construct the Weil representation or the metaplectic representation, a projective representation of the sympletic group which is constructed using interwing operators of this representation of Heisenberg group. 

The first object we need is the sympletic group of $W$, which is the auotmorphism of $W$ that can preserve form $<,>$. The sympletic group takes action on $H(W)$ via $g(w,t)=(g(w),t)$. And this action is trivial on the center of $H(W)$.

\begin{defi}
The metaplectic group:
\begin{equation*}
    \tilde{Sp}_{\phi}(W):=\{(g,w_{\phi}):\rho_{\phi}(gw,t)\circ w_{\phi}(g)=w_{\phi}(g)\rho_{\phi}(w,t),\ for\ \forall (w,t )\in H(W)\},
\end{equation*}
in which $(\rho, \mathcal{S})$  is the representation of Heisenberg group. By Stone-Von Neuman, $W_{\phi}(g)$ is unique up to scalar.
\end{defi}
\begin{rem}
For the metapletic group, we have the short exact sequence:
\begin{equation*}
  0\rightarrow \mathbb{C}^{\times} \rightarrow \Tilde{Sp}_{\phi}(W)\xrightarrow{p}Sp(W)\rightarrow 0  
\end{equation*}
$\tilde{Sp}_{\phi}(W):=Mp(W)$ is the metapletic group that we need. For this group, we have the projection map as 
\begin{align*}
      \tilde{Sp}_{\phi}(W)&\rightarrow Aut(W)\\
      (g, w_{\phi}(g))&\rightarrow w_{\phi}(g).
\end{align*}
\end{rem}
\begin{thm}
The projection map $p$ that restricts to the commutator subgroup $\hat{Sp}_{\phi}(W)=[\tilde{Sp}_{\phi}(W),\tilde{Sp}_{\phi}(W)]$ is a map onto $Sp_{\phi}(W)$ with a kernel of order 2:
\begin{equation*}
    \tilde{Sp}_{\phi}(W)=\hat{Sp}_{\phi}(W)\times \mathbb{C}^{\times}. 
\end{equation*}
2-sheeted covering $\hat{ Sp}_{\phi}(W)$ of $Sp(W)$ is independent of the additive character $\phi$ but Weil representation restricted to $\hat{Sp}_{\phi}(W)$ does so. 
\end{thm}

Now we need to review the Schr\"odinger model of metaplectic representation over a local field $F$. $W=W_{1}\bigoplus W_{2}$ is the complete polarisation of $W$. 

Now $Sp(W)$ can be written as matrices with respect to the basis $\{e_{1},...,e_{n},f_{1},...,f_{n}\}$, $e_{i}\in W_{1}$, $f_{i}\in W_{2}$ and $<e_{i},f_{j}>=\delta_{ij}$.

The first case is 
$\begin{pmatrix}
A & 0\\
0 & ^{t}A^{-1}
\end{pmatrix}\in Sp(W)$. The action of this matrix is metaplectic representation on $\mathcal{S}(W_{1})$ is given by 
\begin{equation*}
    w_{\phi}
    \begin{pmatrix}
    A & 0\\
    0 & ^{t}A^{-1}
    \end{pmatrix}f(x)=|\text{det}(A)|^{\frac{1}{2}}f(^{t}Ax)
\end{equation*}
in which the $|\text{det}A|^{\frac{1}{2}}$ makes action of $GL(W_{1})$ unitray for Hermitian structure on $\mathcal{S}(W_{1})$.

The second case is $\begin{pmatrix}
1 & B\\
0& 1
\end{pmatrix}\in Sp(W)$ if and only if $B=^{t}B$, the action of this matrix is 
\begin{equation*}
    w_{\phi}\begin{pmatrix}
    1 & B\\
    0 &1
    \end{pmatrix}f(x)=\phi(^{t}xBx)f(x).
\end{equation*}

And the third case acts like $w_{\phi}\begin{pmatrix}
0 & 1 \\
-1 & 0
\end{pmatrix}f(x)=\gamma \hat{f}(x)$, in which $\gamma$ is the 8th root of unity and 
\begin{equation*}
    \hat{f}(x)=\int_{k^{n}} f(y)\phi(\sum^{n}_{i=1}x_{i}y_{i})dy
\end{equation*}
and $\hat{\hat{f}}(x)=f(-x)$.

\subsubsection{Stone-Weierstrass theorem}
At the end of this section of Weil representation, we recall the Stone-Weierstrass theorem \cite{repka1978stone}. 

Let's introduce some notations first. Denote $G$ as a separable locally compact group. And given $(\sigma, \mathbb{H})$) as a representation of $G$, where $\mathbb{H}$ is a Hilbert space. $\psi$ is denoted a continuous square-integrable function on $G$. Given $U$ as a open set and $\psi(U)\not=0$, then $L^{2}(U)\subset L^{2}(G)$ is a closed subspaces consisting of functions which vanishes off $U$.

\begin{lem}\label{lem4}
Given $A$ an algebra of continuous functions on $G$ which vanishes at infinity and separate points. $A$ is closed under complex conjugation. 

Then $\Psi\circ A=\{\psi\circ f:f\in A\}$ is dense in $L^{2}(U)$.
\end{lem}
\begin{lem}\label{lem5}
Suppose the intersection of $L^{2}(G)$ with the set of coefficients functions of $\sigma$ spans a dense subspace of $L^{2}(G)$. Then up to isomorphism the regular representation of $G$ is quasi-contained in $\sigma$, i.e. contained in a direct sum of copies of $G$.
\end{lem}
\begin{rem}
The proof of these two lemmas is skipped. The detail about these two lemmas is in \cite{repka1978stone}.
\end{rem}
\begin{prop}
Given $G$ a separable locally group and $(\pi, V)$ a faithful representation of $G$, for finite prime $p$, the coefficient function of $\pi$ belongs to $L^{p}(G)$ and the set of coefficient functions with such property vanishes at infinity. Then the regular representations of $G$ is quasi-contained in 
\begin{equation*}
    \sigma=\otimes'_{m,n\geq 0}\pi_{m,n}=(\pi\otimes\pi\otimes...)\otimes (\tilde{\pi}\otimes\tilde{\pi}\otimes...)\ \ \ \ m\ \text{copies of}\ \pi\ \text{and}\ n\ \text{copies of}\ \tilde{\pi}.
\end{equation*}
\end{prop}
\begin{proof}
By \ref{lem5}, we only need to find the set of coefficient functions that is dense in $L^{2}(G)$. Now given $\phi\in L^{p}(G)$ as a coefficient function of $\pi$, there exists $n$ such that $\phi^{n}\in L^{2}(G)$ and 
$\phi^{n}$ is the coefficient function of $\pi_{n,0}$. $A$ is the algebra of coefficient functions and $\phi^{n}\circ A=\{\phi^{n}\circ f:f\in A\}$. $\sigma$ contains $\pi_{n,0}\otimes \sigma$. By \ref{lem4}, $A\cap L^{2}(G)$ contains the dense subset of $L^{2}(U)$ and $A\cap L^{2}(G)$ is invariant under the translation of $G$. For any $g\in G$, $A\cap L^{2}(G)$ contains the dense subset of $L^{2}(g\cdot U)$. Because $U$ is open, $L^{2}(G)$ can be written as finite sum of functions belonging in $L^{2}(g\cdot U)$. If choosing an appropriate $g\in G$, this approximation can be rewritten by $A\cap L^{2}(G)$ which means any function belonging in $L^{2}(G)$ can be written as summation of functions from $A\cap L^{2}(G)$.
\end{proof}

\subsection{The Construction of Middle Action and the Properties}
First consider $S(GL_{2}(F))$ 
 as bi-$G$-module with fixed Haar measure $\text{d}^{\times}(g)$ to make the volume of the maximal compact subgroup is 1. We need to understand the spectral decomposition of this bi-module $S(GL_{2}(F))$ by using Plancherel measure for $GL_2(F)$ \cite{arthur1991local}:
 For a pair of irreducible representations $(\pi,V)$ and $(\pi',V')$ of $GL_{2}(F)$, if $V\cong V'$, we have  
 \begin{equation*}
\tilde{V}\otimes_{G}S(GL_{2}(F))\otimes_{G}V'\xrightarrow{\cong}\mathbb{C},
 \end{equation*}
otherwise,
\begin{equation*}
\tilde{V}\otimes_{G}S(GL_{2}(F))\otimes_{G}V'=0
\end{equation*}
where $\tilde{V}$ is the contragradient of $V$ taken as a right module, which means every bi-$G$-module $V\otimes_{G}\tilde{V}$ appears uniquely up to isomorphism. We skip the discussion about the dependence of this decomposition on the continuous part of the spectrum. 

The main theme that we need to discuss is the spectral decomposition of $S(M_{2}(F))$, which relates to the construction of Godement-Jacquet L-functions. The aspect we are going to consider is compute the spaces: $\tilde{V}\otimes_{G}S(M_{2}(F))\otimes_{G}V$ with irreducible generic representation $(\pi, V)$. As considering about the diagonal part of the spectral decomposition, we observe that the two actions of Bernstein part of $G$ on $S(M_{2}(F))$ coincide because the two actions of the center on $S(G)$ coincide, as well as that $S(M_{2}(F))$ embeds in the contragradient of $S(G)$. This means that the off-diagonal components $\tilde{V}\otimes_{G}S(M_{2}(F))\otimes_{G}V'$ with $V\not=V'$ are 0. 
This can be shown using the fact that the elements of the Bernstein center of $G$ must commute with the action of $G$. Using this fact, one can indeed show that the tensor product $\widetilde{V} \otimes_G S(M_2(G)) \otimes_G V’$ is $0$ for $V\neq V’$ generic. Indeed, the Bernstein center of $G$ acts on $V’$ with one character, and on $V$ with another. However, because it commutes with $S(M_2(G)$, it must act on both sides of it via the same character, unless the tensor product is actually $0$.
















































Denote $Y=S(M_{2}(F)\times F^{\times})$. There are already two $G$-actions, then we need to use Weil representation to construct the middle action. This middle action is defined as the following theorem.
\begin{thm}\label{thm2.2}
There exists a third action on $Y$, and this action is defined uniquely by three properties:
\begin{enumerate}
    \item This middle action is commutative with the right and left actions;
    \item Given generic representation $(\pi, V)$, there is a $G-$module isomorphism:
    \begin{equation*}
        \nu:\tilde{V}\otimes_{G}Y\otimes_{G}V\rightarrow\mathcal{K}(V(-1))(1)
    \end{equation*}
    \item The isomorphism is compatible with the commutative diagram:
\begin{center}
    \begin{tikzcd}
\tilde{V}\otimes_{G}Y^{0}\otimes_{G}V \arrow[rr, "\mu"] \arrow[d] &  & S(F^{\times}) \arrow[d] \\
\tilde{V}\otimes_{G}Y\otimes_{G}V \arrow[rr, "\nu"]               &  & \mathcal{K}(V(-1))(1)            
\end{tikzcd}
\end{center}
in which $Y^{0}:=S(GL_{2}(F)\times F)$.
\end{enumerate}
\end{thm}
\subsubsection{The Construction of Middle Action}
To prove this theorem, we construct this middle action then prove the three properties. 
The idea is that we need to find the suitable metaplectic group via constructing the symplectic space to make $M_{2}(F)$ lagrangian subspace. This first step is to understand what $\tilde{V}\otimes_{G}Y\otimes_{G}V$ is, in which $Y=S(M_{2}(F)\times F^{\times})$. Here we consider generic representation $(\pi, V)$, i.e. has a Whittaker model. We can take the representation $(\pi, V)$ as a left $G-$module by the left action: $g\cdot v=\pi(g)v$ and the contragradient representation $(\tilde{\pi}, \tilde{V})$ of $(\pi, V)$ has a left action of $G$, which can make it into a right action by $g\rightarrow g^{-1}$.

 $\tilde{V}\otimes_{G}S(M_{2}(F))\otimes_{G}V$ is a 1-dimensional vector space. It's not interesting enough, to avoid the redundancy of proof, we need to consider  $\tilde{V}\otimes_{G}Y\otimes_{G}V$, which is because Godement-jacquet Zeta integral is invariant under translation: $g\rightarrow hgh'$.

Now Let's admit the existence of middle action. Then $Y$ becomes a $G\times G\times G$-module. For the notation in the theorem \ref{thm2.2}, $\tilde{V}\otimes_{G}Y\otimes_{G}V$ can be well defined as a result of tensor product of $ G$-modules. And by constructing this notation, we want to cancel the two trivial left and right action of $G$ on $Y$, then make $\tilde{V}\otimes_{G}Y\otimes_{G}V$ $G$-module with the action of Weil representation.



First, we take these two product step by step. Next, we need to turn one of actions into right action. 
\begin{equation*}
    \tilde{V}\otimes_{G}Y := \tilde{V}\otimes Y/<g\cdot\tilde{v}\otimes y-\tilde{v}\otimes g_{r}\cdot y>,
\end{equation*}
in which $g_{r}$ means the right action of $G$ on $Y$. After canceling out the action on $\tilde{V}$ and the action of $g_{r}$. Now $\tilde{V}\otimes_{G}Y$ is a $G\times G$-module.
\begin{equation*}
    (\tilde{V}\otimes_{G}Y)\otimes_{G} V:= (\tilde{V}\otimes_{G}Y)\otimes V/<g_{l}\cdot(\tilde{v}\otimes_{G} y)\otimes v-\tilde{v}\otimes_{G} y\otimes g\cdot v>,
\end{equation*}
in which the $\tilde{v}\otimes_{G} y$ is defined as the representative of quotient module. And in order to inherit the left action of $Y$ into $\tilde{V}\otimes_{G}Y$, we require that
\begin{equation*}
    g_{l}\cdot(\tilde{v}\otimes_{G} y)=\tilde{v}\otimes_{G} g_{l}\cdot y.
\end{equation*}
Taking those two tensor product, $\tilde{V}\otimes_{G}Y\otimes V$ becomes a simple $G$-module with the action of Weil representation.

Except taking tensor product one by one, we can take $Y$ as $G\times G^{2}$-module, in which $G^{2}:=G\times G$. And $\tilde{V}\boxtimes V$ is a $G\times G$-module with separate actions from the representation of $G$. $\tilde{V}\otimes V$ is a representation of $\Delta G\subset G^{2}$.
\begin{equation*}
    Y\otimes_{G^{2}} (\tilde{V}\boxtimes V)=\tilde{V}\otimes_{G} Y\otimes_{G} V
\end{equation*}
By using this box tensor product, we can directly get the $G$-module that we need.

On the other hand, we can also justify notation of definition of tensor product of modules. 
\begin{rem}
For the $G$-modules i.e. the modules over group ring $\mathbb{Z}[G]$, they form an Abelian category $G-Mod$.
\end{rem}
For group ring $\mathbb{Z}[G]$, and fixed the modules $\tilde{V},\ Y,\ V$ as above, the tensor product $\tilde{V}\otimes_{G}Y$ of modules over $\mathbb{Z}[G]$ is an Abelian group with a balanced product (more information about this check in appendix \ref{balanced})
\begin{equation*}
    \otimes: \tilde{V}\times Y\rightarrow \tilde{V}\otimes_{G}Y,
\end{equation*}
which is universal in the following sense:
\begin{center}
\begin{tikzcd}
\tilde{V}\times Y \arrow[rr, "\otimes"] \arrow[rrd, "f"] &  & \tilde{V}\otimes_{G} Y \arrow[d, "\tilde{f}", dashed] \\
                                                         &  & \mathcal{A}                                          
\end{tikzcd}
\end{center}
For every Abelian group $\mathcal{A}$ and balanced product $f:\tilde{V}\times Y\rightarrow \mathcal{A}$, there is a unique homomorphism $\tilde{f}: \tilde{V}\otimes_{G}Y\rightarrow \mathcal{A}$, such that $\tilde{f}\circ \otimes=f$. 

The tensor product can also be defined as a representative objector for the functor $\mathcal{A}\rightarrow L_{\mathbb{Z}[G]}(\tilde{V}, Y, \mathbb{Z}[G])$(the definition of this notation is in the appendix \ref{balanced}). Explicitly, there is an isomorphism:
\begin{align*}
\text{Hom}_{\mathbb{Z}}(\tilde{V}\otimes_{G}Y, \mathcal{A})&\xrightarrow{\cong} L_{\mathbb{Z}[G]}(\tilde{V}, Y, V)\\
g&\rightarrow g\circ \otimes.
\end{align*}
  
For $\tilde{v}\in \tilde{V}$ and $y\in Y$, one can write $\tilde{v}\otimes y$ for the image of $(\tilde{v},y)$ under $\otimes: \tilde{V}\times Y\rightarrow \tilde{V}\otimes Y$. Strictly, the more accurate notation is $\tilde{v}\otimes_{G}y$, since $G$ is fixed, we drop $G$. From the definitio of balanced product:
\begin{enumerate}
    \item $\tilde{v}\otimes (y+y')=\tilde{v}\otimes y+\tilde{v}\otimes y'$;
    \item $(\tilde{v}+\tilde{v'})\otimes y=\tilde{v}\otimes y +\tilde{v'}\otimes y$;
    \item $g\cdot\tilde{v}\otimes y=\tilde{v}\otimes g\cdot y$.
\end{enumerate}

For several modules, the binary tensor product is associative. So 
\begin{equation*}
    (\tilde{V}\otimes_{G}Y)\otimes_{G}V\cong \tilde{V}\otimes_{G}(Y\otimes_{G} V).
\end{equation*}

So the construction of $\tilde{V}\otimes_{G}Y$ takes a quotient of a free Abelian group with basis the symbols $\tilde{v}*y$, used here to denote the ordered pair $(\tilde{v}, y)$, for $\tilde{v}$ in $\tilde{V}$ and $y$ in $Y$ by the subgroup generated by all elements of the form
\begin{enumerate}
    \item $-\tilde{v}*(y+y')+\tilde{v}*y+\tilde{v}*y'$;
    \item 
    $-(\tilde{v}+\tilde{v'})+\tilde{v}*y+\tilde{y}*y$;
    \item $(g\cdot \tilde{v})*y-\tilde{
    v}*(g\cdot y)$,
\end{enumerate}
where $\tilde{v},\ \tilde{v'}\in \tilde{V}$, $y,y'\in Y$ and $g\in G$. The quotient map which takes $\tilde{v}*y=(\tilde{v}, y)$ to the coset containing $\tilde{v}*y$; That is 
\begin{equation*}
    \otimes : \tilde{V}\times Y\rightarrow \tilde{V}\otimes_{G} Y,\ \ (m,n )\rightarrow [m*n], 
\end{equation*}
is balanced.

So far based on the construction of tensor product of $G-$modules, we can see the actions from those two $G$-modules have been quotiented out, which can be same as the coinvraiant version of tensor product.













Other task is to explain the notation $\mathcal{K}(V(-1))(1)$.  For a representation $V$, the notation $V(n)$ stands for the twist of $V$ by the character $|\text{det}(g)|^n$.
The notation $\mathcal{K}(V)$ is just the Kirillov model of $V$, which is isomorphic to $V$.
So $\mathcal{K}(V(-1))(1)$ is still isomorphic to the original representation $V$ as a representation. The only difference is that it is realized in a slightly different way from the usual Kirillov model. That is, as a space of functions on $F^\times$, given non-trivial additive character $e:F\rightarrow  \mathbb{C}^{\times}$ the mirabolic group acts on it via:
$\begin{pmatrix} 
a & b \\ 
0 & 1 
\end{pmatrix}$
sending the function $f(y)$ to:
$|a| e(bx) f(ay)$
instead of $e(bx) f(ay)$.


Let's get back to the theorem \ref{thm2.2}. For the isomorphism in property (3), it's a canonical isomorphism defined as:
\begin{align*}
    \mu: \tilde{V}\otimes_{G}Y^{0}\otimes_{G}V&\rightarrow S(F^{\times})\\
    \tilde{v}\otimes_{G}\Phi\otimes_{G}v&\rightarrow \int_{GL_{2}(F)}<\tilde{v},\pi(g)v>\cdot \Phi(g,y\text{det}(g)^{-1})\text{d}^{\times}g.
\end{align*}

Now the preparation of construction of middle action is done. 



The goal is to turn $S(M_{2}(F))$ into a Weil representation of a metapletic group. Then we use the action of this metapletic group to construct the desired middle action. After construction this middle action, the proof of properties of theorem\ref{thm2.2} is evidently. To show that, we need to find appropriate symplectic space to make $M_{2}(F)$ a Lagrangian space. 

Denote $U=M_{2}(F)$, and $
    (m, m')\rightarrow <m ,m'>=\text{tr}(m)\text{tr}(m')-\text{tr}(mm')$
is the polarization of the quadratic map $m\rightarrow \text{det}(m)$. Given a two dimensional vector space $W$ with a standard basis $<e_{1},e_{2}>$, $U\otimes W$ can be a symplectic space via 
\begin{equation*}
    <m\otimes v, m'\otimes v'>=(v\wedge v')\cdot <m, m'>=(v\wedge v')\cdot (\text{tr}(m)\text{tr}(m')-\text{tr}(mm')).
\end{equation*}
 
So the symplectic group is 
\begin{equation*}
    Sp(U\otimes W)=Sp(M_{2}(F)\otimes W)=Sp(8, F).
\end{equation*}
In particular, we have a map:
    $G^{3, \text{det}=1}\rightarrow Sp(U\otimes W)$
in which $$G^{3,\text{det}=1}=\{(g_{1},g_{2},g_{3})\in G_{3}:\text{det}(g_{1}g_{2}g_{3}=1)\}.$$
And this embedding is given as $(g_{1}, g_{2},g_{3})\cdot (m\otimes v)= g_{1}m
    g_{3}^{T}\otimes g_{2}v.$
\begin{rem}
$G^{3, \text{det}=1}$ acts on $U\otimes W$ symmetrically. Identify $U=M_{2}(F)$ as the endomorphism group of $W$ such that $M_{2}(F)=\text{End}(W)=W\otimes W^{*}$, where $W^{*}$ denotes the dual space of $W$. Then $U\otimes W=W\otimes W^{*}\otimes W$, which makes the symplectic form invariant. In a other word, we can describe  this symmetrical embedding as 
$S_{3}\ltimes G^{3, \text{det}=1}\rightarrow Sp(U\otimes W)$.

And this gives a symmetry of the three actions on $Y$, in which the middle will be constructed later. So we can consider $Y$ takes actions of $S_{3}\ltimes G^{3}$. In particular, the action of transpose $(1,3)\in S_{3}$ on $Y$ is given by 
\begin{equation*}
    (1,3)\cdot \varrho(m,t)=\varrho(g^{T},t).
\end{equation*}

Recall that $g\rightarrow g^{-T}$ is the Cartan invollution on $G$, given by the inverse of the transpose  map $g^{-T}=(g^{T})^{-1}$.
\end{rem}







\begin{rem}
The embedding above can canonically lift to the double cover $Mp(U\otimes W)$.
\begin{center}
% https://tikzcd.yichuanshen.de/#N4Igdg9gJgpgziAXAbVABwnAlgFyxMJZABgBpiBdUkANwEMAbAVxiRAHEA9YAZlIB1+OGAA8cwWDgC+AXgCMUkFNLpMufIRQAmclVqMWbAMpoAFAFVBEPAFt4AAgDqASiUqQGbHgJEdcvfTMrIggALJmlvzWWHZwTq5SejBQAObwRKAAZgBOEDZIZCA4EEgK7jl5SDpFJYhlWbn5iIXFVdQMWGDBIFB0cAAWySDtdABGMAwACmremiAMMJk4w-pBbII2dDj9AMaMwGiKiVJAA
\begin{tikzcd}
{G^{3,\text{det}=1}} \arrow[rr] \arrow[rrd, "p", dashed] &  & Sp(U\otimes W)           \\
                                                                   &  & Mp(U\otimes W) \arrow[u]
\end{tikzcd}
\end{center}
And we have the exact sequence:
\begin{equation*}
    1\rightarrow \mu_{2}\rightarrow Mp(U\otimes W)\rightarrow Sp(U\otimes W)\rightarrow 1.
\end{equation*} 
The action of $Mp(U\otimes W)$ is clear by Schr\"{o}dinger model. If we want to make the action of $G^{3 ,\text{det}=1}$ explicit, the map $p$ has to be a group homeomorphism, which is because $Mp(U\otimes W)$ splits over $G^{3, \text{det}=1}$. So we can compose the action of $Mp(U\otimes W)$ on $S(M_{2}(F))$ to get the action of $G^{3, \text{det}=1}$ on $S(M_{2}(F))$.


\end{rem}
Now let's prove there is a unique metaplectic life for the map $S_{3}\ltimes G^{3, \text{det}=1}\rightarrow Sp(U\otimes W)$.
\begin{proof}
First thing to do is to show the existence of this lift. And the first step to prove is that the subgroup $SL_{2}(F)^{3}\subset G^{3,\text{det}=1}$ admits a unique lift. Then we prove the rest of the group $G^{3,\text{det}=1}$. 

Then for the lift of each copy of $SL_{2}(F)$ is unique because its abelianization is trivial Note that the subgroup $SL_{2}(F)\times\{1\}\times SL_{2}(F)$ has metaplectic lift by the following action on $S(M_{2}(F))$:
\begin{equation*}
    (g_{1},1,g_{3})\cdot \Phi(m)=\Phi(g_{1}^{-1}mg^{-1}_{3}).
\end{equation*}
$SL_{2}(F)\times SL_{2}(F)\times\{1\}$ and $\{1\}\times SL_{2}(F)\times SL_{2}(F)$

\textcolor{red}{XXXXXXXXXXXXXXXXXXXXXXXXXXXXXXXXXXXXXXXXXXXXXXXXXXXXXXXXXXXXXXXXXXXx}
\end{proof}


We can start on the construction of Middle action:
Consider $M_{2}(F)=M_{2}(F)\otimes e_{2}$ as a Lagrangian space of $U\otimes W$, which means it's a maximal isotropic subspace, i.e. is a sub-vector space on which the symplectic form vanishes. Then by the Schr\"{o}dinger model of Weil representation of $Mp(U\otimes W)$ on $S(M_{2}(F))$ corresponding to the character $e: F\rightarrow \mathbb{C}^{\times}$, we get an action of 
\begin{equation*}
    G^{3,\text{det}=1}=(GL_{2}(F)\times GL_{2}(F)\times GL_{2}(F))=\{(g_{1}, g_{2}, g_{3}): \text{det}(g_{1} g_{2} g_{3})=1\}.
\end{equation*}
on $S(M_{2}(F))$.
Next, we need to use this model to construct the additional middle on $Y=S(M_{2}(F)\times F^{\times})$. By using compact induction to extend the action of $G^{3,\text{det}=1}$ of $S(M_{2}(F)\otimes e_{2})$ to the action of $G^{3}$ of $Y$, we can identify
\begin{equation*}
    Y=S(M_{2}(F)\times F^{\times})=cInd^{S_{3}\ltimes G^{3}}_{S_{3}\ltimes G^{3,\text{det}=1}}S(M_{2}(F)),
\end{equation*}
via section 
\begin{center}
    % https://tikzcd.yichuanshen.de/#N4Igdg9gJgpgziAXAbVABwnAlgFyxMJZABgBoBGAXVJADcBDAGwFcYkQBxAPWAGZSAOgJwwAHjmCwcAXwC85OQDEeQvAFt40kNNLpMufIRQAmCtTpNW7bny0692PASKnj5hizaJOPfqrESUnIK2rogGI6GRGTE7pZeIACeoQ4GziaksTQeVt4AFOSkAASJxeQAlNrmMFAA5vBEoABmAE4QakhkIDgQSAphre19ND1IxvYggx2IXaOI4wNt0-zdvYgALBNTSCtzxFtLSOsjawqU0kA
\begin{tikzcd}

{G^{3,\text{det}=1}=F^{\times}} \arrow[rr] \arrow[rrd] &  & G^{3} \arrow[d]       \\
                                                       &  & {G^{3,\text{det}=1}} 
\end{tikzcd}
\end{center}
by sending $y\mapsto \bigg(1,\begin{pmatrix}
1 &y\\
0&1
\end{pmatrix}, 1\bigg)$. 
And this is the middle action of $GL_{2}(F)$ on $Y$ that we desire. 

But the first question is what is this representation of $G^{3,\text{det}=1}$ over $S(M_{2}(F))$.


Next we need to show two things: first we prove that as vector spaces, $Y=S(M_{2}(F)\times F^{\times })$ is the space of the induced representation and commutes with the left and right actions on both sides. Second, so far $Y$ is known as a $G\times G$-module. But $cInd^{S_{3}\ltimes G^{3}}_{S_{3}\ltimes G^{3,\text{det}=1}}S(M_{2}(F))$ is a $G\times G\times G$-module. We explain  this tri-module structure later.   In order to introduce the middle action on $Y$, we use the extra action from Weil representation on $cInd^{S_{3}\ltimes G^{3}}_{S_{3}\ltimes G^{3,\text{det}=1}}S(M_{2}(F))$.


Taking this representation of $G^{3,\text{det}=1}$ over $S(M_{2}(F))$ as $G\times G$-module, it should have two actions as following: given $g\in G$ and $\Phi\in S(M_{2}(F))$,
\begin{align*}
     \textbf{Left-action on $S(M_{2}(F))$:}\ \ (g\cdot \Phi)(x)&=\Phi(xg)\\
     \textbf{Right-action on $SM_{2}(F)$:}\ \ (g\cdot \Phi)(x)&=\Phi(xg^{-1})
\end{align*}
To avoid confusion, let us clarify the terminology here: the left action induced by right multiplication and the right action induced by right multiplication. 
Let us denote the representation of $G^{3, \text{det}=1}$ as $(\sigma, S(M_{2}(F)))$, then the induced representation is 
\begin{align*}
       cInd^{S_{3}\ltimes G^{3}}_{S_{3}\ltimes G^{3,\text{det}=1}}S(M_{2}(F)):=\big\{f:G\rightarrow S(M_{2}(F)): f(hg)=\sigma(h)f(g)&\big\}
\end{align*}
And for $f\in G$, we require following properties:    
\begin{enumerate}
    \item There exists a compact open subgroup $K$ of $G$, for every $k\in K$ and $g\in G$, $f(kg)=f(g)$;
    \item The support of $f$ has compact image in $G^{3, \text{det}=1}\backslash G^{3}$.
\end{enumerate}
For the left hand side, $S(M_{2}(F)\times F^{\times})= S(M_{2}(F))\otimes S(F^{\times})$. Now we are ready to establish the isomorphism above: Given $\Phi\otimes \Psi\in S(M_{2}(F))\otimes S(F^{\times})$ and $f\in   cInd^{S_{3}\ltimes G^{3}}_{S_{3}\ltimes G^{3,\text{det}=1}}S(M_{2}(F))$,
\begin{align*}
    \Phi\otimes \Psi \mapsto f(g)(x)=f(g_{1},y)(x)=(g,\Phi)(x)\Psi(y),
\end{align*}
in which $g\in G$ and $y\in F^{\times }$. Since $\Psi\in S(F^{\times})$, $\Psi(y)\in \mathbb{C}$ is a scalar.

Explicitly, we can write the formula of this action by Weil representation: Given $\varrho(m, t)=\Phi(m)\cdot \Psi(t)\in S(M_{2}(F)\times F^{\times})$, then the action of $G^{3}$ on $Y$ is defined as
\begin{flalign*}
    \big((g_{1},g_{2},g_{3})\cdot\varrho\big)(m,t)&=    |\text{det}(g_{1}g_{2}g_{3})|\bigg(\big(g_{1},\begin{pmatrix}
    y & 0\\
    0 & 1
    \end{pmatrix}g_{2}\begin{pmatrix}
    y^{-1}\text{det}(g_{1}g_{2}g_{3})^{-1}&0\\
    0&1
    \end{pmatrix},g_{3}\big)\Phi\bigg)(m)\\
    &\times \Psi(t\cdot\text{det}(g_{1}g_{2}g_{3})),
\end{flalign*}
in which the action of $\bigg(\big(g_{1},\begin{pmatrix}
    y & 0\\
    0 & 1
    \end{pmatrix}g_{2}\begin{pmatrix}
    y^{-1}\text{det}(g_{1}g_{2}g_{3})^{-1}&0\\
    0&1
    \end{pmatrix},g_{3}\big)$ on $\Phi$
is the action obtained from Weil representation. We call the resulting action of the middle copy of $G$ on $Y$ as the middle action on $Y$.
\subsubsection{The Proof of Theorem \ref{thm2.2}}
After finishing the construction, we need to introduce some facts before the proof of theorem \ref{thm2.2}.

By the character $e$ above, the corresponding character $\theta$ of $U_{2}(F)\subset GL_{2}(F)$ of upper triangular
unipotent matrices is defined as follow:
\begin{equation*}
    \theta: U\rightarrow \mathbb{C}^{\times}\ \ \ \ \theta\bigg(\begin{pmatrix}
    1 & u\\
    0 & 1
    \end{pmatrix}\bigg)=e(u).
\end{equation*}
The mirabolic subgroup $P$ of $G=GL_{n}(F)$ is defined as $P=\begin{pmatrix}
a&b\\
0&1
\end{pmatrix}$, and because we are considering the case of $GL_{2}$, it's much easier: $G_{1}=P_{1}=U=\begin{pmatrix}
1 & a\\
0& 1
\end{pmatrix}.$ Denote $\mathfrak{Rep}(G)$ as the category of smooth representation of $G$, $\mathfrak{Rep}(P)$ as the category of representation of $P$, $\mathfrak{Rep}(G_{1})$ as the representation of $G_{1}$ and $\mathfrak{Rep}(P_{1})$ as the representation of $P_{1}$. In order to be more prise with the representation of groups, we will use the notations from  I N Bernshtein and A V Zelevinskii \cite{bernstein1976representations}. Given $(\pi, P, E)$ as a representation of subgroup $P_{2}$, then there exists a functor $\Phi^{-}$ which sends $(\pi,P,E)$ to $(\Phi^{+}(\pi),P_{1}, E/E(U,\theta))$. and if given a representation $(\tau, P_{1}, V)$, then $\Phi^{+}(\tau)=Ind^{P}_{P_{1}}\tau$ as induced representation of $P_{2}$, which we denotes as $Ind(P, P_{1}, \tau')$ and it acts as $\tau'(p\cdot u)\xi=\theta(u)\tau(p)\xi$.

\textcolor{red}{XXXXXXXXXXXXXXXXXXXXXX}

So we have a two functors as following:
\begin{enumerate}
    \item $\Phi^{-}:\mathfrak{Rep}(P)\rightarrow \mathfrak{Rep}(P_{1}): (\pi,P,E)\rightarrow (\Phi^{-}(\pi),P_{1}, E/E(U,\theta))$;
    \item $\Phi^{+}:\mathfrak{Rep}(P_{1})\rightarrow \mathfrak{Rep}(P_{2}):(\tau, P, V)\rightarrow Ind(P_{2},P_{1},\tau')$.
\end{enumerate}
And there are other two functors between categories $\mathfrak{Rep}(P)$ and $\mathfrak{Rep}(G_{1})$. The functor $\Psi^{-1}$ sends $(\pi, P_{2},E)$ to $\Psi^{-1}(\pi)$ which representation space is $E_{U,1}=E/<\pi(u)\xi-\xi>$. Given a representation $(\rho,G_{1},V)$, then $(\Psi(\rho),P,V)$ is a representation of $P$, in which $\Psi(\rho)(gu)=\rho(g)$ for $g\in G,\ u\in U$.
\begin{enumerate}
    \item $\Psi^{-}:\mathfrak{Rep}(P)\rightarrow \mathfrak{Rep}(G_{1}):(\pi, P, E)\rightarrow (\Psi^{-1}(\pi),G_{1},E_{U,1})$;
    \item $\Psi^{+}:\mathfrak{Rep}(G_{1})\rightarrow \mathfrak{Rep}(P):(\rho, G, V)\rightarrow (\Psi^{+}, P, V)$.
\end{enumerate}
\begin{prop}
For the properties of these four functors $\Phi^{+},\ \Phi^{-},\ \Psi^{+}, \ \Psi^{-}$:
\begin{enumerate}
    \item All these four functors are exact;
    \item $\Phi^{+}$ is the left adjoint of $\Phi^{-}$, i.e. for all any $\pi\in\mathfrak{Rep}(P)$ and $\tau\in \mathfrak{Rep}(P_{1})$ then there exists isomorphism 
    \begin{equation*}
        \text{Hom}_{P}(\Phi^{+}(\tau),\pi)=\text{Hom}_{P_{1}}(\tau,\Phi^{-}(\pi)),
    \end{equation*}
    Similarly, $\Psi^{+}$ is the left adjoint of $\Psi^{-}$, i.e. 
    \begin{equation*}
        \text{Hom}_{P}(\pi,\Psi^{+}(\rho))=\text{Hom}_{G}(\Psi^{-}(\pi),\rho);
    \end{equation*}
    \item Based on the isomorphisms, we can have following morphisms: 
\begin{align*}
    i: \Phi^{+}\Phi^{-}(\pi)\rightarrow  \pi & i':\tau\rightarrow \Phi^{-}\Phi^{+}(\tau)\\
    j: \pi\rightarrow \Psi^{+}\Psi^{-}(\pi)& j':\Psi^{-}\Psi^{+}(\rho)\rightarrow \rho,
\end{align*}
in which 
\begin{align*}
    \Phi^{+}\Phi^{-}:\mathfrak{Rep}(P)\rightarrow \mathfrak{Rep}(P) & \Phi^{-}\Phi^{+}:\mathfrak{Rep}(P_{1})\rightarrow \mathfrak{Rep}(P_{1})\\
    \Psi^{+}\Psi^{-}: \mathfrak{Rep}(P)\rightarrow \mathfrak{Rep}(P) & \Psi^{-}\Psi^{+}: \mathfrak{Rep}(G_{1})\rightarrow \mathfrak{Rep}(G_{1})
\end{align*}
then there exists a exact sequence:
\begin{equation*}
    0\rightarrow \Phi^{+}\Phi^{-}(\pi)\xrightarrow{i}\pi \xrightarrow{j} \Psi^{+}\Psi^{-}(\pi)\rightarrow 0
\end{equation*}
\end{enumerate}

\end{prop}
\begin{rem}\label{BZ}
In the proof of this theorem, it follows that the representation $\pi^{0}=\mathcal{E}_{c}(Y)$ is isomorphic to the restriction of $\pi$ to $E(M,1)$. And we know that 
\begin{equation*}
    \phi^{+}\phi^{-}(\pi)\cong Ind(P_{2},P_{1},\phi^{-}(\pi)')\cong\pi^{0}
\end{equation*}
For the rest of the proof, it's in \cite{bernstein1976representations}.

\end{rem}

\begin{thm}\label{1dim}
Given a generic, irreducible, admissible representation $(\pi, V)$ of $G=GL_{2}(F)$, then $\text{dim} V_{U,\theta}=1$ for any  character $\theta$, in which $V_{U, \theta}$ is defined as 
\begin{equation*}
    V_{U,\theta}=V/V(U,\theta)=V/<\pi(u)\xi-\theta(u)\xi)>,\ \ (u,\xi)\in U\times V.
\end{equation*}
\end{thm}
The proof of the this theorem is from \cite{bernstein1976representations}.



\begin{lem}
Let $Y$ be a $G^{3}$-module via left, right and middle actions respectively. Then the resulting three actions of Bernstein centers of each copy of $G$ on Y identify.
\end{lem}
\begin{rem}
For the definition and basic information about Bernstein center, check appendix \ref{Bernstein}.
\end{rem}
\begin{proof}
For the left and right translation actions of $G$ on $S(M_{2}(F))$ because the $G\times G$-equivariant pairing:
\begin{align*}
    S(M_{2}(F))\otimes S(G)&\rightarrow  \mathbb{C}\\
    \Psi \otimes f &\mapsto \int_{G} \Psi(g)f(g)\text{d}^{\times}g
\end{align*}
is non-degenerate and because the Bernstein center of $G$ acts on $S(G)$ the same way from both sides. As the result, the left and right action of the center of $G$ on $Y$ coincide. By symmetry (i.e. by conjugation by a permutation from $S_{3}$), this also applies to the middle action.
\end{proof}
\begin{lem}
The map $\Phi^{+}\Phi^{-}Y\rightarrow Y$ is an embedding with the image $Y^{0}=S(GL_{2}(F)\times F^{\times})\subset Y$, where the functor $\Phi^{+}\Phi^{-}$ is taken with the respect to the middle action on $Y$.
\end{lem}
\begin{proof}
The middle action of the matrix $\begin{pmatrix}
a &b\\
0&1
\end{pmatrix}$ on the element $\Phi(g, y)\in Y$ is:
\begin{equation*}
\begin{pmatrix}
a &b\\
0&1
\end{pmatrix}\cdot \Phi(g, y)=|a|\cdot e(b\text{det}(g)y)\cdot \Phi(g, ay).
\end{equation*}
By \ref{BZ}, it follows that the image in lemma consists of the functions supported away from the set $\{\text{det}(g)y=0\}$.
\end{proof} 

Now we are ready to prove the theorem \ref{thm2.2}:
\begin{proof} 
We have finished the action of middle action of $G$ on $Y$. And we know that it's commutative with the left and right actions of $G $ on $Y$.

The first statement we are proving is the existence of the isomorphism $\nu$ of theorem \ref{thm2.2}.

The space $\Phi^{-}\tilde{V}$ is one-dimensional by \ref{1dim}. A choice of vector $\mathbb{C}\rightarrow \Phi^{-}\tilde{V}$ gives a map of $P-$modules by adjunction:
\begin{equation*}
    S(F^{\times})\cong \Phi^{+}\mathbb{C}\rightarrow \tilde{V}.
\end{equation*}
Taking  the dual map, we obtain a morphism of $P-$modules: $V\rightarrow\widetilde{S(F^{\times})}$, where $\widetilde{S(F^{\times})}$ is identified with the space of smooth functions on $F^{\times}$, and $P$ acts on $\widetilde{S(F^{\times})}$ by:
\begin{equation*}
    \begin{pmatrix}
    a&b\\
    0&1
    \end{pmatrix}\cdot f(y)=e(by)f(ay).
\end{equation*}
The map $V\rightarrow \widetilde{S(F^{\times})}$ is the non-zero map of $P-$modules from $V\rightarrow $ to $\widetilde{S(F^{\times})}$, which is unique up to scalar. These maps are injective and have the same image. This image is called the Kirillov model $\mathcal{K}(V)$ of $V$. The related notation Whittaker model of $V$ is obtained by taking the map $S(F^{\times})\rightarrow \tilde{V}$ of $P-$modules and extending it to a map of $G-$modules: $\mathbb{1}_{Y}\rightarrow\tilde{V}$ by making $\mathbb{1}_{Y}$ the induction with compact support of $S(F^{\times})$ from $P$ to $G$. The space $\mathbb{1}_{Y}$ is called the Whittaker space, and the image of the resulting dual map $V\rightarrow\tilde{\mathbb{1}_{Y}}$ is the Whittaker model of $V$.




First, we consider $(\pi, V)$ as irreducible principal series representation and supercuspidal representation. 


\end{proof}

\begin{lem}
Let $(\pi, V)$ be a representation of $G$, such that $V$ and $\tilde{V}$ admits no $SL_{2}(F)$-invariant linear functionals. Then 
\begin{equation*}
    \tilde{V}\otimes_{G}S(M^{\text{det}=0}_{2}(F)\times F^{\times})\otimes_{G} V =\text{Jac}(\tilde{V})\otimes_{F^{\times}\times F^{\times}} \text{Jac}(V).
\end{equation*}
\end{lem}
\subsubsection{The Properties of the Middle Action}
\appendix
\section{Bernstein Center}\label{Bernstein}
Before the story begins, let's explain why we need this categorically abstract definition of a center. However, the Bernstein center of $G$ is much richer! It turns out that it contains enough operators that, given a generic representation of $G$, that representation can be fully characterized by the action of the Bernstein center on it (in fact, in this case, the Bernstein center is fully known).

 First, let us set so notations. Given $F$ as a Non-Archimedean local field and denote $G=GL_{2}(F)$. Then we pick a compact and open subgroup $K$ of $G$. $e_{K}=\chi(K)/\text{vol}(K)$ is the characteristic funtion of $K$. $\hat{G}$ denotes as the set of irreducible smooth representation of $G$ over $\mathbb{C}$. $H(G)$ is the Hecke algebra, which is the convolution algebra of locally constant $\mathbb{C}$-valued functions on $G$ with compact support. 
 \begin{cla}
 \begin{equation*}
     H(G)= \varinjlim_{K} H(G, K)
 \end{equation*}
 \end{cla}
 \begin{defi}
 We define the Bernstein center $Z(G)$ abstractly as the endomorphism ring of the identity functor of the category of smooth complex representaion $\mathfrak{G}$ of $G$. Then the Bernstein center $Z(G)$ acts on any smooth representation and this action commutes with any $G-$morphism. 
 \end{defi}
 \begin{rem}
 More precisely, $Z(G)=\text{End}(Id_{\mathfrak{G}})$. For an element $\phi\in Z(G)$ is a set of maps $\phi_{A}: A\rightarrow A$ for $A\in \text{ob}(\mathfrak{G})$ such that for $\alpha: A\rightarrow B$, the following diagram commutes:
 \begin{center}
\begin{tikzcd}
A \arrow[rr, "\phi_{A}"] \arrow[d, "\alpha"] &  & A \arrow[d, "\alpha"] \\
B \arrow[rr, "\phi_{B}"]                     &  & B                    
\end{tikzcd}
 \end{center}
 \end{rem}
\begin{ex}
 For example, given an algebra, the Bernstein center of the category of modules for that algebra is the center of the algebra (i.e., elements that commute with all others). In particular, the Bernstein center of $G$ contains the actual center (as a group) $Z(G)$ of $G$.
\end{ex}
 Next, our goal is to describe the abstract words in the definition of Bernstein center $Z(G)$.
 
 \begin{defi}
 Denote the center of $H(G, K)$ by $Z(G, K)$ and let 
 \begin{equation*}
     \hat{H}(G)=\varprojlim_{K} H(G, K),\ \ \ \ Z(G)=\varprojlim_{K} Z(G, K),
 \end{equation*}
 \end{defi}
 where the transition maps are given by applying idempotents(i.e.  $f\in H(G, K)\mapsto e_{K'}*f*e_{K'}$ for $K'\subset K$).  
 \newpage
\section{Monoidal category}
A monoidal category is a category $\mathfrak{C}$ equipped with 
\begin{enumerate}
    \item functor $\otimes: \mathfrak{C}\times \mathfrak{C}\rightarrow \mathfrak{C}$ out of the product category of $\mathfrak{C}$ with itself;
    \item an object $\mathbb{1}\in \mathfrak{C}$ unit object or we call this as the tensor unit;
    \item a natural isomorphism $a:((-)\otimes(-))\otimes(-)\cong(-)\otimes((-)\otimes(-))$ with components of the form $a_{x,y,z}: (x\otimes y)\otimes z\rightarrow x\otimes(y\otimes z)$;
    \item a natural isomorphism $\lambda: (\mathbb{1}\otimes(-))\cong(-)$ with components of the form $\lambda_{x}:\mathbb{1}\otimes x\rightarrow x$, which is called the left unit;
    \item a natural isomorphism $\rho:((-)\otimes\mathbb{1})\rightarrow (-)$ with components of the form $\rho_{x}:x\otimes\mathbb{1}\rightarrow x$, which is called the right unit.
\end{enumerate}
such that the following two kinds of diagrams commute, for all involved:
\begin{enumerate}
    \item The triangle identity:
\begin{center}
    % https://tikzcd.yichuanshen.de/#N4Igdg9gJgpgziAXAbVABwnAlgFyxMJZABgBpiBdUkANwEMAbAVxiRAAoAPAHW4jwC28XgLo4AFgCNJwAIwBfAJS9+WIXAAEATxDzS6TLnyEUAJnJVajFmx59B8drJUPNWxbv0gM2PASKypLKW9MysiCB2quraupYwUADm8ESgAGYAThACSGQgOBBIgVZhbHQA+sCcpBoiYlIyCjVa8p7pWTmIxQVI5iU2EXUS0nLylZzyLmrwtdwMdAKSUBXALW0gmdm51D2IfaEDILwZ4hDjk-bTmkMNo5Vr8hTyQA
\begin{tikzcd}
(x\otimes\mathbb{1})\otimes y \arrow[rr, "{a_{x, \mathbb{1}, y}}"] \arrow[rd, "\rho_{x}\otimes \mathbb{1}_{y}"] &            & x\otimes(1\otimes y) \arrow[ld, "\mathbb{1}_{x}\otimes \lambda_{y}"] \\
                                                                                                                & x\otimes y &                                                                     
\end{tikzcd}
\end{center}
\item The pentagon identity:
\begin{center}
    % https://tikzcd.yichuanshen.de/#N4Igdg9gJgpgziAXAbVABwnAlgFyxMJZARgBoAGAXVJADcBDAGwFcYkQAKAdwB0eI8AW3gACAB4BKPgKzC4IjgE9pQ0QC8JIAL6l0mXPkIoATKWLU6TVu24rZojmLtyFy-qvkaJmnXux4CIlNjCwYWNkQQXnd7eQ5HZ1FFKRiXDW1dEAx-QyJyUhCaMOtI21SHJ3L5ZJSZNIy-A0CUfPMiqwjOMrrRSUTq2o8RNW0LGCgAc3giUAAzACcIQSQAFhocCCR8y3D2egB9YGie+TFSEUVSNS0GkAWlrfXNxDIdkpADo-Ozi-7hm98d0Wy0QphAGyQr2KnQAklBDlwtH8+Ew0AALT4-S7XW73EFrcHPADM7V2kU+XFIZ0USKqIjhhxxgLxSBJhKQYOhewR3z+2IBlC0QA
\begin{tikzcd}
                                                                                                            & (w\otimes x)\otimes (y\otimes z) \arrow[rd, "{a_{w, x, y\otimes z}}"] &                                                                               \\
((w\otimes x)\otimes y)\otimes z \arrow[ru, "{a_{w\otimes x, y,z}}"] \arrow[d, "{a_{w,x,y}\otimes Id_{z}}"] &                                                                       & (w\otimes (x\otimes (y\otimes z)))                                            \\
(w\otimes (x\otimes y))\otimes z \arrow[rr, "{a_{w, x\otimes y,z}}"]                                        &                                                                       & w\otimes ((x\otimes y)\otimes z) \arrow[u, "{Id_{w}\otimes \alpha_{x, y,z}}"]
\end{tikzcd}
\end{center}
\end{enumerate}
\section{Balanced Product}\label{balanced}
As we said earlier, for a given $G$-module, we can take it as a module over group ring $\mathbb{Z}[G]$. So for ring $\mathbb{Z}[G]$, a  right module $\tilde{V}$, a module $Y$ with the three actions, and a left module $V$. Given an arbitrary Abelian group $\mathcal{A}$, for example, we deal with $\tilde{V}$ and $Y$ first, a map $\Gamma_{\mathcal{A},\mathbb{Z}[G]}: \tilde{V}\times Y\rightarrow \mathcal{A}$ is said to be a $\mathbb{Z}[G]$-balanced, $\mathbb{Z}[G]$-middle-linear or $\mathbb{Z}[G]$-balanced product, if for $\tilde{v},\ \tilde{v'}\in \tilde{V}$, $y,\ y'\in Y$ and $g\in \mathbb{Z}[G]$ the following holds:
\begin{enumerate}
    \item $\Gamma_{\mathcal{A},\mathbb{Z}[G]}(\tilde{v}, y+y')=\Gamma_{\mathcal{A},\mathbb{Z}[G]}(\tilde{v},y)+ \Gamma(\tilde{v},y')$;
    \item $\Gamma_{\mathcal{A},\mathbb{z}[G]}(\tilde{v}+\tilde{v'},y)=\Gamma_{\mathcal{A}+\mathbb{Z}[G]}(\tilde{v},y)+\Gamma_{\mathcal{A},\mathbb{Z}[G]}(\tilde{v'},y)$;
    \item $\Gamma_{\mathcal{A},\mathbb{Z}[G]}(g\cdot \tilde{v},y)=\Gamma_{\mathcal{A},\mathbb{Z}[G]}(\tilde{v},g\cdot y)$.
\end{enumerate}
The actions in the definition are introduced in earlier section. And
the set of all such maps as balanced product are denoted as $L_{\mathbb{Z}[G]}(\tilde{V},Y,\mathcal{A})$.

And given two balanced product $\Gamma_{1},\ \Gamma_{2}$, the operations $\Gamma_{1}+\Gamma_{2}$ and $-\Gamma$ are defined pointwisely. Those two operations make $L_{\mathbb{Z}[G]}(\tilde{V},Y,\mathcal{A})$ an Abelian group.
\newpage
\bibliographystyle{plain}
\bibliography{bib.bib}


\end{document}
